\section{Impact and benefits of the project}
\label{sec:impact}

\Comments{
Describe the dissemination and/or exploitation strategy envisaged.
Describe how the results of the project address an ANR 2018 work programme challenge or theme ; specify in what field(s) (scientific, economic, social or cultural) the results of the project may have an impact.
In the case of a Young Researchers project (JCJC), 
Specify the project's capacity to promote the development of a topic or the young researcher's own team.
}

\Comments{- L'adéquation du projet n'est pas suffisamment justifiée par rapport à l'axe et le défi portés par ce comité. 
- L'impact du projet manque de précision tant au niveau scientifique qu'applicatif.
}

\subsection{pre-proposition}
%instrument JCJC
%Defi 7, axe4, secondairement axe1. 

This project combines both a local characterization of digital surfaces 
and a toolkit development for the analysis of interfaces in 3D digital images, 
possibly degraded with noise. 
As a consequence, it fits to the principal component ``Research and innovation''
of the 2018 work program of the French National Research Agency 
and more specifically to the 7th societal challenge ``Information and Communication Societies''
of the generic call (B7).    
In this challenge, we target the research theme $\sharp$4 about data, knowledge, multimedia content, AI, 
because the call text explicitly mention fundamental research projects in digital image processing in this theme. However, some aspects of the project (\wpPattern) are part of the research theme $\sharp$1, 
which is complementary to the theme ``Mathematics''.    

We also target the young researcher funding instrument (\emph{JCJC}) 
to let the PI work in a small team of experts, strengthened by students, so that
he develops an innovative research theme introducing arithmetic into geometry processing and
reaps the benefits of its preliminary work \cite{LPRJMIV2017}. 
%\todo{+++}


%impacts scientifiques
This work was not the main goal but a side task of previous collaborative projects: 
DigitalSnow ANR-11-BS02-009 and CoMeDiC ANR-15-CE40-0006. 
We think that it is now time to completely address this issue 
and investigate the new perspectives that the proposed solutions open. 
This project has therefore a scientific impact at least as wide as the one of the above-mentioned projects. 
%impacts eco, socio, culturels
More precisely, accurate estimations (\wpEstim)
will impact computer graphics (surface fairing, volume rendering)
and 3D image analysis (geometric measurements, surface deformation), 
whereas noise detection (\wpScale) may become a key step of 3D image processing pipelines.  
Thus, this project may have long-term impacts on industries that use, produce or sell such software tools. 

%montrer qu'on va gagner en efficacite, rentabilite dans le future
%DGtal, DGtalTools
In order to ease the use of our methods in applications for the future, we will make them available in the open-source C++ library \DGtal~and in the open-source mathematical software system \sage~(\wpPattern). \DGtal~is a powerful diffusion tool because it is common in the digital geometry community and become known by other communities since its software award of the Symposium on Geometry Processing in 2016. 
%publier 
We will strive to publish our results in top venues in digital geometry or geometry processing, 
in peer-reviewed international journals or in \IPOL, which promotes reproducible research in image processing 
and provides an online demonstration facility. 
