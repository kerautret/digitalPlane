\section{Impact and benefits of the project}
\label{sec:impact}

\Comments{
Describe the dissemination and/or exploitation strategy envisaged.
Describe how the results of the project address an ANR 2018 work programme challenge or theme ; specify in what field(s) (scientific, economic, social or cultural) the results of the project may have an impact.
In the case of a Young Researchers project (JCJC), 
Specify the project's capacity to promote the development of a topic or the young researcher's own team.
}

\Comments{- L'adéquation du projet n'est pas suffisamment justifiée par rapport à l'axe et le défi portés par ce comité. 
- L'impact du projet manque de précision tant au niveau scientifique qu'applicatif.
}

%dissemination strategry
%publier
\noindent\textbf{Dissemination strategy.}
PARADIS aims at proposing a new framework for the analysis of digital surfaces.
As such, it is a fundamental research project with expected results in digital
geometry, combinatorics on words, geometry processing and computer vision. Hence,
its first outcomes will be publications in highly ranked international journals,
\eg Transactions on Image Processing (TIP), Computer Vision and Image Understanding (CVIU),
Journal of Mathematical Imaging and Vision (JMIV), Theoretical Computer Science (TCS), etc.
In order to have a broad impact, we will target top conferences gathering different communities,
\eg the Eurographics Symposium on Geometry Processing (SGP),
the Conference on Computer Vision and Pattern Recognition (CVPR),
but also smaller and very specialized events such as the Conference
of Discrete Geometry for Computer Imagery (DGCI) or the Conference on Words (Words). 
We believe that this communication strategy will sow the seeds for future collaboration
beyond the project.

To enhance the dissemination, publications of preprints in open access archive
will also be considered. For instance, Elsevier, \eg CVIU, and IEEE, \eg TIP,
allow for the free access of the author's versions of the preprint.
Wiley, publisher of Computer Graphics Forum (SGP), permit the authors to share their pre-print after
a 12-month embargo period. A similar policy applies to Springer, \eg JMIV,
Lectures Notes on Computer Sciences (DGCI, Words). The Computer Vision Foundation
automatically gives open access to papers published in major vision conferences, such as CVPR. 

Another strategy is to target open journals. {\IPOL} is a peer-reviewed journal, which promotes
reproducible research in image processing and 3D. Note that D. Coeurjolly, B. Kerautret and J-O. Lachaud
belong to the editorial board of {\IPOL}. B. Kerautret has also been the editor of a special issue on
digital geometry in 2011 and has made {\DGtal} be usable on the {\IPOL} plateform for this occasion. 
We plan to release our last software pieces both as a {\DGtal} tool and an {\IPOL} paper
(see \sect{sec:wp} and fig.~\ref{fig:gantt}).

%montrer qu'on va gagner en efficacite, rentabilite dans le future
%DGtal, DGtalTools
More generally, we will make our methods available in {\DGtal} 
%the open-source C++ library {\DGtal}
and in the open-source mathematical software system {\sage}
in order to both facilitate the collaborations between the team members
and easily reuse our works for future applications.
Note that the current main editors of {\DGtal} are {D. Coeurjolly} and
{J.-O. Lachaud}, while {B. Kerautret} and the PI are active developers.
S. Labb\'{e} is the main developer of the \texttt{sage/combinat/words} package
and has its own optional package \texttt{slabbe}.

New contributions to the geometry package of
{\DGtal} are planned as two deliverables: one includes an algorithm for
digital plane segment recognition at T+24, the other includes core classes
for a multiscale analysis of digital surfaces at T+48. A tool will be build
upon theses bricks and added to {\DGtalTools}, a companion project of
{\DGtal} (see \sect{sec:wp} and fig.~\ref{fig:gantt}).  

%TODO dire qu'on est capable, et que les communautes sont ouvertes a ces travaux ?

The PI has already published related papers in digital geometry and computer vision,
\eg \cite{LPRTCS2016,LPRDGCI2016,LPRJMIV2017}. 
Yet, the PI and its collaborators have not published so much in geometry processing until now,
\eg \cite{Coeurjolly2017}.
The SGP software award that has been given to {\DGtal} in 2016 indicates that the
geometry processing community is open to relevant and interesting work about digital surfaces
and that the team has the potential to publish in these venues.


%- L'impact du projet manque de précision tant au niveau scientifique qu'applicatif.
\noindent\textbf{Scientific impact.}
PARADIS aims at proposing new geometric estimators on digital surfaces and
specifically seeks a parameter-free and adaptive normal estimator.
%interet des normales
As for point-clouds or meshes (\sect{sec:art}), numerous high-level tasks
rely on the quality of the normal estimation on digital surfaces: rendering,
surface fairing, surface deformation for physical simulation or tracking,
scene understanding, precise geometric measurements, primitive extraction, etc. 
%interet du adaptatif et sans parametres
In this project, we search for an estimator that is (i) efficient to compute,
(ii) parameter-free and (iii) accurate, \ie it preserves sharp features and
is multigrid-convergent. These characteristics are very important for the
applications. Accuracy is an indispensable condition to guarentee the final
results and being parameter-free is a feature that is often sought by users,
which prefer tools working out of the box.

As a consequence, this project will benefit all high-level tasks based on a
normal field onto a digital surface. In \ref{wp2}, we plan to focus on applications
that not only rely on a normal field but also on position information:
polyhedral approximation, surface fairing and rendering. Among many other
applications, we detail below two examples related to previous collaborative projects: 
DigitalSnow ANR-11-BS02-009 and CoMeDiC ANR-15-CE40-0006. PARADIS is indeed based
based on preliminary works, which were not the main goal but side tasks of these projects.

DigitalSnow aimed at modeling snow metamorphism from 3D images of real snow
microstructures acquired using X-ray tomography techniques (fig.~\ref{fig:snow}).
In this project, a normal field is required to move the ``ice-air'' interface
in the normal direction at a velocity that depends on physical parameters,
on the geometry or both. A normal field provides also a way of measuring 
the faceting effect observable on snow grains under a strong temperature gradient.  

The CoMeDiC project aimed at filling the gap between standard calculus and
discrete calculus, which is a powerful framework for solving discrete variational
problems in image and geometry processing, for subsets of the digital space $\Z^n$, 
such as digital surfaces. In this project, a normal field is estimated to define
well-chosen metrics for discrete calculus that make it converge toward continuous values.

We think that it is now time to completely address these issues and investigate
the new perspectives that the proposed solutions open. 
PARADIS has therefore an impact at least as wide as the one of the above-mentioned
projects. 

More precisely, accurate estimations \ref{wp2} will impact computer graphics
(surface fairing, rendering) and 3D image analysis (geometric measurements, surface deformation), 
whereas noise detection \ref{wp3} may become a key step of 3D image processing pipelines.  
Thus, PARADIS may have long-term impacts on industries that use, produce or sell such software tools. 


%programme challenge/ theme
%Defi 7, axe4, secondairement axe1. 
\noindent\textbf{Positionning.}
PARADIS combines both a local characterization of digital surfaces 
and a toolkit development for the analysis of interfaces in 3D digital images, 
possibly degraded with noise. 
As a consequence, it fits to the principal component ``Research and innovation''
of the 2018 work program of the French National Research Agency 
and more specifically to the 7th societal challenge ``Information and Communication Societies''
of the generic call (B7).    

%% le traitement de grandes masses de données produites par l’observation
%% scientifique en biologie, en physique, en astrophysique, etc. le calcul intensif pour la simulation
%% dans la plupart des disciplines, les objets connectés pour l’observation scientifique, etc

%% Le défi « Société de l'information et de la communication » s'inscrit ainsi dans une
%% double priorité : penser le numérique au service de la société et concevoir et développer le
%% numérique de demain via l'évolution de concepts, de méthodes et d'outils. Il s’articule en 7
%% axes :

In this challenge, we target the research theme $\sharp$4 about data, knowledge, multimedia content, AI, 
because the call text explicitly mention fundamental research projects in digital image processing in this theme. However, some aspects of the project (\wpPattern) are part of the research theme $\sharp$1, 
which is complementary to the theme ``Mathematics''.    



%instrument JCJC
%development of a topic
We also target the young researcher funding instrument (\emph{JCJC}) 
to let the PI work in a small team of experts, strengthened by students, so that
he develops an innovative research theme introducing arithmetic into geometry processing and
reaps the benefits of its preliminary work \cite{LPRJMIV2017}. 



