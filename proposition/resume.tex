\documentclass[french]{article}
 
\usepackage[utf8]{inputenc}
\usepackage[T1]{fontenc}
\usepackage{babel}

\begin{document}

%Résumés scientifiques (non confidentiel) en français ou en anglais (au moins un des
%deux résumés doit être renseigné, 100 à 1000 caractères).
%TRIS en fait les 2 sont obligatoires...


%TRIS: nouvelle version pour faire moins de 1000 caractères 
This project focuses on the geometry of digital surfaces, which are boundaries of voxel sets. These data mainly come from the segmentation of 3D digital images. Keeping the digital nature of the data is often an advantage. However, a drawback is its poor geometry at any resolution. The challenge is to enhance its geometry by estimating extra data for each surface element, such as a relevant normal vector.
The idea is to gather the geometrical information around each surface element within a patch of adaptive size: a piece of digital plane that locally fits to the digital surface. The covering of a digital surface by maximal pieces of digital plane is however hard because of their combinatorial explosion. An opportunity to make a breakthrough in this issue is the recent development of plane-probing algorithms. Based on these algorithms, we propose a new way of analyzing digital surfaces without any parameter. We expect a positive impact in graphics and 3D image analysis.

%This project focuses on the processing of digital surfaces, which are boundaries of voxel sets. These data come from the segmentation of 3D digital images or are created from scratch by voxel editors. Keeping the digital nature of the data is an advantage for several tasks. However, a drawback is its poor geometry at any resolution. The challenge is to enhance its geometry by estimating extra data for each surface element, such as a relevant normal vector.  

%Since we are looking for first-order estimations, we will gather the geometrical information around each surface element within a patch of adaptive size: a piece of digital plane that locally fits to the digital surface. The covering of a digital surface by maximal pieces of digital plane is however hard because of their combinatorial explosion. An opportunity to make a breakthrough in this issue is the recent development by the principal investigator of plane-probing algorithms. The project will be based on this class of algorithms.

% JOL
%This new way of analyzing 3D digital data and estimating its linear geometry does not require any parameter. We expect that such new tools can have a positive impact in several applications of 3D image analysis (geometric measurements, 3d noise estimation and removal) and of computer graphics (surface fairing, volume rendering, polyhedral approximation).

~

%TODO traduction
Ce projet porte sur la géométrie des surfaces discrètes délimitant des ensembles de voxels. Ces données viennent surtout de la segmentation d'images 3D. Conserver la nature discrète des données est un souvent avantage. Cependant, la géométrie est pauvre à toute résolution. Le défi est d'augmenter la géométrie de données supplémentaires à chaque élément de surface, comme un vecteur normal pertinent.
L'idée est de résumer l'information géométrique se trouvant autour de chaque élément de surface dans une zone adaptée : un morceau de plan discret qui s'ajuste localement à la surface. Recouvrir une surface discrète par de tels morceaux maximaux est difficile à cause de leur explosion combinatoire. Une opportunité d'avancer sur cette question est le développement récent d'algorithmes exploratoires. Basé sur ces algorithmes, nous proposons d'analyser les surfaces discrètes sans paramètre, ce qui aura des impacts positifs en informatique graphique et analyse d'images 3D.
  

\end{document}