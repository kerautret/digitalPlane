\documentclass[french]{article}
 
\usepackage[utf8]{inputenc}
\usepackage[T1]{fontenc}
\usepackage{babel}

\begin{document}

ENGLISH

~
%%%%%%%%%%
In numerous fields of application, such as material sciences
or medical imaging, non-invasive acquisition devices such as magnetic
resonance, X-ray tomography or micro-tomography are required for
observation, measurements or diagnostic aids.
These acquisition devices usually generate large data sets, i.e.
3D images, composed of regularly spaced data in a cuboidal domain.
3D volumes come from the segmentation of such 3D images.
They are also generated in scientific modelling because
numerous simulation schemes rely on the regularity of the data support.

PARADIS is a project about geometry of volume boundaries,
called digital surfaces. 
Keeping the digital nature of the data is an advantage
to do integer-only and exact computations,
to perform constructive solid geometry operations,
or to use efficient spatial data structures.
A drawback is its poor geometry, because a digital surface is only 
made up of quadrangular surface elements whose normal vector is parallel to one axis, 
whatever the resolution.  
Many tasks in computer graphics, vision and 3D image analysis require a richer geometry: 
rendering, surface deformation for simulation or tracking, precise geometric measurements, etc.
To perform relevant geometric tasks and 
to benefit from the above-mentioned advantages in the same time, 
we need to enhance the geometry of digital surfaces by estimating extra data for each surface element. 
This project focuses on estimations of local and first-order geometric quantities
such as normal vector direction.
It aims at providing accurate and parameter-free estimators
based on a surface patch with adaptive size around each surface element.
Since we are looking for first-order estimations, the patch will be typically a piece of digital plane
that locally fits the digital surface.

A challenge is to cover the whole digital surface by maximal pieces of digital plane. 
Such a cover will not only provide a normal vector field, but will also provide, if computed
for several subsampled versions of the input 3D volume, a way of determining the scale 
at which noise is unlikely: a high number of very small segments indicate noise, whereas
smooth parts are decomposed into a smaller set of larger segments.    

What is challenging is that there is a combinatorial explosion
of maximal pieces of digital plane and that among them,
not all are tangent to the digital surface.  
An opportunity to make a breakthrough regarding this issue is to consider the recent development
of plane-probing algorithms, proposed by the principal investigator and its collaborators.
These algorithms allow to decide
on-the-fly how to probe the digital surface and make grow a digital plane segment,
which is tangent by construction. The growth direction is given by both arithmetic and geometric properties.
PARADIS aims at analyzing digital surfaces with the help of plane-probing algorithms. 

In computer graphics and 3D image analysis, 
numerous high-level tasks applied on digital surfaces 
rely on the quality of normal estimation: 
rendering, surface fairing, surface deformation 
for physical simulation or tracking, scene understanding, 
geometric measurements, primitive extraction, etc. 
In addition, since many 3D images can be degraded with noise, 
especially in medical imaging, noise detection is a crucial task,  
that may become a key step of 3D image processing pipelines.  
Thus, PARADIS may have long-term impacts on industries 
that use, produce or sell such software tools. 

%%%%%%%%%%

~

FRENCH

~
%%%%%%%%%%

%%%%%%%%%%
~

  

\end{document}
